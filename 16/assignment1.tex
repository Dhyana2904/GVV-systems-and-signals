%\iffalse
\let\negmedspace\undefined
\let\negthickspace\undefined
\documentclass[journal,12pt,twocolumn]{IEEEtran}
\usepackage{cite}
\usepackage{amsmath,amssymb,amsfonts,amsthm}
\usepackage{algorithmic}
\usepackage{graphicx}
\usepackage{textcomp}
\usepackage{xcolor}
\usepackage{txfonts}
\usepackage{listings}
\usepackage{enumitem}
\usepackage{mathtools}
\usepackage{gensymb}
\usepackage{comment}
\usepackage[breaklinks=true]{hyperref}
\usepackage{tkz-euclide} 
\usepackage{listings}
\usepackage{gvv}                                        
\def\inputGnumericTable{}                                 
\usepackage[latin1]{inputenc}                                
\usepackage{color}                                            
\usepackage{array}                                            
\usepackage{longtable}                                       
\usepackage{calc}                                             
\usepackage{multirow}                                         
\usepackage{hhline}                                           
\usepackage{ifthen}                                           
\usepackage{lscape}

\newtheorem{theorem}{Theorem}[section]
\newtheorem{problem}{Problem}
\newtheorem{proposition}{Proposition}[section]
\newtheorem{lemma}{Lemma}[section]
\newtheorem{corollary}[theorem]{Corollary}
\newtheorem{example}{Example}[section]
\newtheorem{definition}[problem]{Definition}
\newcommand{\BEQA}{\begin{eqnarray}}
\newcommand{\EEQA}{\end{eqnarray}}
\newcommand{\define}{\stackrel{\triangle}{=}}
\theoremstyle{remark}
\newtheorem{rem}{Remark}
\begin{document}
\bibliographystyle{IEEEtran}
\vspace{3cm}
\title{\textbf{12.10.16}}
\author{EE23BTECH11210-Dhyana Teja Machineni$^{*}$% <-this % stops a space
}
\maketitle
\newpage
\bigskip

\textbf{QUESTION:}
In double-slit experiment using light of wavelength 600 nm, the
angular width of a fringe formed on a distant screen is 0.1°. What is
the spacing between the two slits?

SOLUTION:
Let the equation of the light waves coming from the source be
\begin{align}
y_1=A \sin(2\pi f t)\\
y_2=A \sin(2\pi f t+\phi)
\end{align}
Since both the light waves are from the same source so the frequency of both the waves is same frequency f and amplitude A.

USing principle of superposition, we get
\begin{align}
y &= y_1+y_2
\end{align}
where $y$ is resultant wave equation
\begin{align}
y& = y_1+ y_2\\
y&= A sin(2\pi f t)+ A sin (2\pi f t+\phi)
\end{align}
using
\begin{align}
sin(c)+sin(d)= 2sin(\dfrac{c+d}{2})cos(\dfrac{c-d}{2})
\end{align}
we get
\begin{align}
y&=2Asin(2\pi f t+\frac{\phi}{2})cos(\frac{\phi}{2})
\end{align}
For constructive interference to happen 
\begin{align}
cos(\phi/2)&= +/- 1\\
\phi&= 2 n \pi
\end{align}
Equation(8) is the condition for constructive interference
 In YDSE setup, the path difference between the light rays is given by 
 \begin{align}
 \Delta x&= \frac{\lambda}{2\pi}\phi\\
 \Delta x&= d sin(\theta)
 \end{align}
 from the equations (9) and (10)
 \begin{align}
 \frac{\lambda}{2\pi}2 n \pi= d sin(\theta)
 \end{align}
 \begin{align}
     d sin(\theta)= n\lambda
 \end{align}
 Now, for small values of $\theta$, we can approximate 
\begin{align}
    sin(\theta)&\approx tan(\theta)
    \frac{n}{d}&=\frac{y}{D}
    y&=n\frac{D \lambda}{d}
\end{align}
 Now let us find the fringe width for this interference
 Let fringe width be $\beta$
 \begin{align}
 y_{n+1}&=(n+1)\frac{D\lambda}{d}\\
     y_n&=n\frac{D\lambda}{d}\\
     \beta&= y_{n+1}-y_n\\
     \beta&=\frac{D\lambda}{d}
 \end{align}
 Angluar Fringe width for light rays in YDSE is given by 
 \begin{align}
     Tan(\theta)&=\frac{\beta}{D}
 \end{align}
 For small angles we can assume 
 \begin{align}
     Tan(\theta)&\approx \theta
\end{align}
     From equation (21)
     \begin{align}
     \theta&=\frac{\frac{\lambda D}{d}}{D}\\
     \therefore \theta&=\frac{\lambda}{d}
 \end{align}
 Given
 \begin{align}
     \theta&=0.1\degree\\
&=\frac{\pi}{1800}\\
    \lambda&=600nm
 \end{align}
 \begin{align}
 d&=\frac{600}{\frac{\pi}{1800}}\\
 \therefore d= 3.44*10^{-4}m.
 \end{align}

 \newpage

 \renewcommand{\thefigure}{\theenumi}
 \renewcommand{\thetable}{\theenumi}

 \begin{flushleft}
     \begin{table}[h]
         \caption{Variables and their descriptions}
         \label{tab:table1.12.10.16}
         \renewcommand{\arraystretch}{1.5}
\begin{tabular}{|c|c|c|c|c|c|c|c|c|c|}
\hline
VARIABLE& \textbf{Description}&\textbf{Value}\\\hline
$y_1$& Equation of first wave&none\\\hline
$y_2$& Equation of the second wave&none\\\hline
$k$& Wave number of the light ray&none\\\hline
$f$& Frequency of the light ray&none\\\hline
$y$& Equation of the resultant light wave&none\\\hline
$A$& Amplitude of the light wave&none\\\hline
$f$& Frequency of both the wave equations&none\\\hline
$\Delta x$& Path difference between the light rays&none\\\hline
$\phi$& Phase difference between the light rays&none\\\hline
$\beta$& Fringe width of the interface formed by the light rays&none\\\hline
$D$& Distance between the centre of the slits and the screen&none\\\hline
$d$& Spacing between the slits used in the YDSE&NEED TO BE FOUND\\\hline
$\lambda$& Wavelength of the light used&600nm\\\hline
$\theta$& Angular fringe width&0.1\degree\\\hline
\end{tabular}

     \end{table}
 \end{flushleft}

\end{document}
