%\iffalse
\let\negmedspace\undefined
\let\negthickspace\undefined
\documentclass[journal,12pt,twocolumn]{IEEEtran}
\usepackage{cite}
\usepackage{amsmath,amssymb,amsfonts,amsthm}
\usepackage{algorithmic}
\usepackage{graphicx}
\usepackage{textcomp}
\usepackage{xcolor}
\usepackage{txfonts}
\usepackage{listings}
\usepackage{enumitem}
\usepackage{mathtools}
\usepackage{gensymb}
\usepackage{comment}
\usepackage[breaklinks=true]{hyperref}
\usepackage{tkz-euclide} 
\usepackage{listings}
\usepackage{gvv}                                        
\def\inputGnumericTable{}                                 
\usepackage[latin1]{inputenc}                                
\usepackage{color}                                            
\usepackage{array}                                            
\usepackage{longtable}                                       
\usepackage{calc}                                             
\usepackage{multirow}                                         
\usepackage{hhline}                                           
\usepackage{ifthen}                                           
\usepackage{lscape}

\newtheorem{theorem}{Theorem}[section]
\newtheorem{problem}{Problem}
\newtheorem{proposition}{Proposition}[section]
\newtheorem{lemma}{Lemma}[section]
\newtheorem{corollary}[theorem]{Corollary}
\newtheorem{example}{Example}[section]
\newtheorem{definition}[problem]{Definition}
\newcommand{\BEQA}{\begin{eqnarray}}
\newcommand{\EEQA}{\end{eqnarray}}
\newcommand{\define}{\stackrel{\triangle}{=}}
\theoremstyle{remark}
\newtheorem{rem}{Remark}
\begin{document}
\bibliographystyle{IEEEtran}
\vspace{3cm}
\title{\textbf{12.10.16}}
\author{EE23BTECH11210-Dhyana Teja Machineni$^{*}$% <-this % stops a space
}
\maketitle
\newpage
\bigskip

\textbf{QUESTION:}
Find the sum to indicated number of terms in each of the geometric progressions in
0.15, 0.015, 0.0015, ... 20 terms.
\section*{Solution}

 \renewcommand{\thefigure}{\theenumi}
 \renewcommand{\thetable}{\theenumi}

 \begin{flushleft}
     \begin{table}[h]
         \caption{Variables and their descriptions}
         \label{tab:table1}
         \renewcommand{\arraystretch}{1.5}
\begin{tabular}{|c|c|c|c|c|c|c|c|c|c|}
\hline
VARIABLE& \textbf{Description}&\textbf{Value}\\\hline
$y_1$& Equation of first wave&none\\\hline
$y_2$& Equation of the second wave&none\\\hline
$k$& Wave number of the light ray&none\\\hline
$f$& Frequency of the light ray&none\\\hline
$y$& Equation of the resultant light wave&none\\\hline
$A$& Amplitude of the light wave&none\\\hline
$f$& Frequency of both the wave equations&none\\\hline
$\Delta x$& Path difference between the light rays&none\\\hline
$\phi$& Phase difference between the light rays&none\\\hline
$\beta$& Fringe width of the interface formed by the light rays&none\\\hline
$D$& Distance between the centre of the slits and the screen&none\\\hline
$d$& Spacing between the slits used in the YDSE&NEED TO BE FOUND\\\hline
$\lambda$& Wavelength of the light used&600nm\\\hline
$\theta$& Angular fringe width&0.1\degree\\\hline
\end{tabular}

     \end{table}
 \end{flushleft}
Let \( x(0) \) denote the first term and \( r \) the common ratio. The sum of a geometric progression with \( n \) terms:
\begin{align}
x(n) &= x(0)r^n \\
X(z) &= \frac{x(0)}{1-rz^{-1}} \\
S(z) &= X(z)U(z) \\
     &= \frac{x(0)}{(1-rz^{-1})(1-z^{-1})} \quad \lvert z \rvert > \lvert r \rvert \\
     &= \frac{x(0)(\frac{r}{1-rz^{-1}}-\frac{1}{1-z^{-1}})}{(r-1)}
\end{align}
The inverse of S(z) is s(n) 
\begin{align}
s(n)= x(0)(\frac{r^{n+1}-1}{r-1})
\end{align}
Substitute the values in the above equation
\begin{align}
s(n)&= 0.15*\frac{0.1^{20} -1}{0.1-1}\\
 \therefore s(n)&= \frac{1}{6}[1-0.1^{20}]
\end{align}
\end{document}

